\documentclass[a4paper, 10pt, oneside]{article}
\renewcommand{\familydefault}{\sfdefault}
%---------------------------------------------------------------------------------
%------------------------- Set the variables for the Mainpage here ---------------
%---------------------------------------------------------------------------------
\newcommand{\name}{John Doe}
\newcommand{\email}{john@doe.com}
\newcommand{\osid}{OS-11111}

%---------------------------------------------------------------------------------
%------------------------- Include the list of used packages ---------------------
%---------------------------------------------------------------------------------
\input{packages.tex}

%---------------------------------------------------------------------------------
%------------------------- Create title page -------------------------------------
%---------------------------------------------------------------------------------
\title{{\textbf{\Huge Offensive Security}}\\ Penetration Test Report for\\Internal Lab and Exam}
\author{\vspace{3cm}\\{\LARGE \name}\\[1em]\email\\[1em]OSID: \osid}
\date{\vspace{7cm}\today}


\begin{document}


%---------------------------------------------------------------------------------
%------------------------- Print title page --------------------------------------
%---------------------------------------------------------------------------------
\maketitle
\thispagestyle{empty}
%---------------------------------------------------------------------------------
%------------------------- Print table of contents -------------------------------
%---------------------------------------------------------------------------------
% Attention: Document might need some runs to get all indexes etc right .. usual LaTeX stuff :)
\tableofcontents
\thispagestyle{empty}
\pagebreak

\section{Offensive	Security Lab	and	Exam Penetration	Test	Report}

\subsection{Introduction}
The	 Offensive	 Security Lab	 and	 Exam	 penetration	 test	 report should	 contain	 all	 the	 steps	 taken	 to	
successfully	compromise	machines	both	in	the	exam	and	lab	environments. Accompanying	data	used	in	
both	environments	should	also	be	included,	such	as	PoCs,	custom	exploit	code,	and	so	on.	Please	note
that	this	report	will	be	graded	from	a	standpoint	of	correctness	and	completeness. The	purpose	of	this	
report	is	 to	ensure	 that	 the	 student	 has	a	 full	 understanding	 of	 penetration	 testing	methodologies	as	
well	 as	 the	 technical	 knowledge	 required	 to	 successfully	 achieve	 the Offensive	 Security Certified	
Professional	(OSCP)	certification.

\subsection{Objective}
The	objective	of	this	assessment	is	to	perform	an	internal	penetration	test	against	the	Offensive	Security
Lab	and	Exam	network. The	student	is	tasked	with	following	methodical	approach	in	obtaining	access	to	
the	objective	goals.	This	test	should	simulate	an	actual	penetration	test	and	how	you	would	start	from	
beginning	to	end,	including	the	overall	report. A	sample	page	has	been	included	in	this	document	that	
should	help	you	determine	what	is	expected	of	you from	a	reporting	standpoint. Please	use	the	sample	
report	as	a	guide to	get	you	through	the	reporting requirement	of	the	course.

\subsection{Requirements}
The	student	will	be	required	to	complete	this	penetration	testing	report	in	its	entirety and	to	include	the	
following	sections:
\begin{itemize}
  \item Overall	High-Level	Summary and	Recommendations	(Non-technical)
  \item Methodology	walk-through	and	detailed	outline	of	steps	taken
  \item Each	 finding	with accompanying screenshots,	walk-throughs,	 sample code,	and	 proof.txt file if
applicable.
  \item Any	additional	items	as	deemed	necessary
\end{itemize}

\section{	Report High-Level	Summary}

OS-XXXXX	was	 tasked	with	 performing	an	internal	 penetration	 test in	 the Offensive	Security Labs and	
Exam	network.	An	internal	penetration	test is	a	simulated attack	against	internally	connected	systems.
	
The	 focus	 of	 this	 test	 is	 to	 perform	 attacks,	 similar	 to	 those	 of	 a	 malicious	 entity, and	 attempt to	
infiltrate	Offensive	Security's	internal	lab	systems the	THINC.local domain,	and	the	exam	network. OSXXXXX's	 overall	 objective	 was	 to	 evaluate	 the	 network,	 identify	 systems,	 and	 exploit	 flaws	 while	
reporting the	findings	back	to	Offensive	Security.

While conducting the	 internal	 penetration	 test,	 there	 were	 several	 alarming	 vulnerabilities	 that	were	
identified	 within Offensive	 Security's	 network.	 For	 example,	 OS-XXXXX	 was	 able	 to	 gain	 access	 to	
multiple	machines,	primarily	due	to	outdated	patches	and	poor	security	configurations.		During	testing,	
OS-XXXXX	had	administrative	level	access	 to	multiple	 systems.	All	 systems	were	 successfully	exploited	
and	access	granted.	These	systems	as	well	as	a	brief	description	on	how	access	was	obtained	are	listed	
below:



\begin{itemize}
        \item Target \#1	- Obtained a	low-privilege shell	via	the	vulnerable	web	application	called	'KikChat'.	
Once	in,	access	was	leveraged	to	escalate to	'root'	using	the	'getsystem' command	in	
Meterpreter
\end{itemize}

\section{Report	- Recommendations}

OS-XXXXX recommends	patching	the	vulnerabilities	identified	during	the	penetration	test to	ensure	that	
an	attacker	 cannot	exploit	 these	 systems	 in	 the	 future.	One	 thing	 to	 remember	 is	 that	 these	 systems	
require	 frequent	 patching	 and	 once	 patched,	 should	 remain	 on	 a	 regular	 patch	 program	 in	 order	 to	
mitigate	additional	vulnerabilities	that may	be discovered	at	a	later	date

\section{Report	- Methodologies}
OS-XXXXX utilized a	 widely	 adopted	 approach	 to	 performing	 penetration	 testing	 that	 is	 effective	 in	
testing	how	well	the	Offensive	Security Labs	and	Exam	environments	are	secure.	Below	is	a	summary of	
how	OS-XXXXX	was	able	to	identify	and exploit	a	number	of	systems.




%---------------------------------------------------------------------------------
%------------------------- Declaring the empty variables for the hosts -----------
%---------------------------------------------------------------------------------
\newcommand{\hostname}{}
\newcommand{\ip}{}
\newcommand{\tcpports}{}
\newcommand{\udpports}{}
\newcommand{\os}{}
\newcommand{\vuln}{}
\newcommand{\product}{}
\newcommand{\vulnx}{}
\newcommand{\productx}{}

%---------------------------------------------------------------------------------
%------------ Here are the hosts, ------------------------------------------------
%------------ just add in the same scheme after you coppied the example folder ---
%---------------------------------------------------------------------------------
\input{./hosts/example/host.tex}
\input{./hosts/DeepThought/host.tex}

\end{document}
